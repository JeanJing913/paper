\documentclass{article}

\usepackage{graphics}
\usepackage{graphicx}
\usepackage{algorithmic,algorithm}
\usepackage{color}
\usepackage{amsthm}
\usepackage{enumerate}
\newtheorem{mydef}{Definition}
\newtheorem{mylm}{Proposition}

\section{The ADMSM Algorithm}
\label{sec:am}

The whole ADMSM algorithm consists of two stages: the preprocessing
stage and the matching stage. In the preprocessing stage, the whole
pattern set is compressed into a compact data structure called
\emph{adaptive match tree} (\emph{AMT}), in which each tree node has
its own structure. Then in the matching stage, the algorithm uses the
\emph{AMT} to match against the text string one character at a time,
and output the matching result once there is a match.

\subsection{Building the adaptive match tree}
\label{sec:pp}

In the preprocessing stage, the pattern set will be transformed to a
\emph{AMT}. As we will see later, each node in \emph{AMT} can have its
own structure according to the characteristic of the pattern set, and
this means the \emph{AMT} can . The
\emph{AMT} is constructed by recursively cutting and grouping the
prefixes of the patterns.  Firstly, the length of the shortest pattern
in the pattern set is computed



\subsection{Matching the text}
\label{sec:mp}

\end{document}