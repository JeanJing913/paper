\documentclass[legalpaper]{article}

\synctex=1 
\usepackage{tikz}
\usepackage{algorithm}
\usepackage{algorithmic}
\usepackage{multirow}
\usepackage{amsmath}

\newcommand{\SWITCH}[1]{\STATE \textbf{switch} (#1)}  
\newcommand{\ENDSWITCH}{\STATE \textbf{end switch}}  
\newcommand{\CASE}[1]{\STATE \textbf{case} #1\textbf{:} \begin{ALC@g}}  
\newcommand{\ENDCASE}{\end{ALC@g}}  
\newcommand{\CASELINE}[1]{\STATE \textbf{case} #1\textbf{:} }  
\newcommand{\DEFAULT}{\STATE \textbf{default:} \begin{ALC@g}}  
\newcommand{\ENDDEFAULT}{\end{ALC@g}}  
\newcommand{\DEFAULTLINE}[1]{\STATE \textbf{default:} }
\newtheorem{mydef}{Definition}
\newtheorem{mylm}{Proposition}

\renewcommand{\algorithmicrequire}{\textbf{Input:}}
\renewcommand{\algorithmicensure}{\textbf{Output:}}

% \title{An Efficient Adaptive Algorithm for Dictionary Matching}
% \author{Peng Zhan \and Wang Yuping}
% \data{}
% \maketitle

\begin{document}
\section{Notations}
\label{sec:stucture}

Let $\Sigma$ be an \textsf{alphabet} consists of a finite number of
character symbols. (In this paper we will mainly focus on the case
that $\Sigma$ is the ASCII character set, where $|\Sigma| = 256$ and
each character requires only one byte of memory.) Given the alphabet
$\Sigma$, a \textsf{string} as well as its \textsf{substring} over
$\Sigma$ can be defined as:\\
\\
\textbf{Definition 1.} A \textsf{string} of length $n$ over $\Sigma$ is
denoted by $S = c_1c_2..c_n$, where $S[i] = c_i \in \Sigma$ $(1 \leq i
\leq n)$, and the length of $S$ is denoted by $|S|$. The
\textsf{substring} of $S$ which starts at position $i$ and has
length $len$ is denoted by $S[i,\,len]$.\\
\\
Another important concept is the \textsf{prefix}, which is a special
substring of a given string:\\
\\
\textbf{Definition 2.} The length-$m$ \textsf{prefix} of $S$ denoted by
$S^{+m}$ is the substring that consists of the first $m$ characters of
$S$, that is $S^{+m}=S[1,\,m]$.  While the notation $S^{-m}$ is used to
denote the substring left after discarding the first $m$ characters of $S$, that is $S^{-m} = S[m+1,\,|S|-m]$.\\
\\
The \textsf{multi-string matching} problem can be formally defined as follows: \\
\\
\textbf{Definition 3.} Given a text string $T$ and a pattern set
$P=\{p_1,\,p_2,\,\dots,\,p_k\}$ where $p_i$ is a string over $\Sigma$.
The \textsf{multi-string matching} is to find all occurrences of every
pattern of $P$ in $T$, more specifically, to find the set $R = \{(i,\,
p_j)\;|\; p_j \in P\; and\; T[i,\,|p_j|]=p_j\}$.


\section{The ADMSM Algorithm}
\label{sec:ADSM}

The running process of the ADMSM algorithm consists of two main
stages: the preprocessing stage and the matching stage. In the
preprocessing stage, the whole pattern set is transformed to a compact
tree-like structure called \textsf{Adaptive Match Tree}
(\textsf{AMT}), where each node in the tree is built adaptively. While
in the matching stage, the built \textsf{AMT} is compared against the
text string to search patterns in that string.  The transformation of
the pattern set to the \textsf{AMT} is introduced in Section
\ref{subsec:preprocessing}, while using the \textsf{AMT} to search
patterns in the text string is stated in Section
\ref{subsec:matching}, and finally the various kinds of tree node
structures are explored in Section \ref{subsec:nodes}.

\subsection{Building the AMT}
\label{subsec:preprocessing}

In the preprocessing stage, the whole pattern set will be transformed
to an \textsf{AMT}. This transformation includes four major steps:

\begin{enumerate}
\item The $lsp$ of the pattern set is computed first, and then all the
length-$lsp$ prefixes of the patterns in the pattern set are cut off
to form a prefix set with the repetitions removed.
\item Create an index structure (actually a tree node) to hold the
prefixes in the prefix set. Each prefix serves as a key in the index
structure. The type of the index structure is selected adaptively
based on the characteristic of the prefix set.
\item Divide the pattern set left from step 1 into sub-pattern-sets:
the patterns with the same length-$lsp$ prefix cutting off are grouped
together to form a sub-pattern-set. Associate each sub-pattern-set with
the corresponding prefix key in the tree node created in step 2.
\item For each newly created sub-pattern-set in step 3, repeat the
same procedure from step 1.
\end{enumerate}

\begin{figure}[htbp]
  \label{fig:AMT}
  \centering
  \includegraphics[width=\textwidth]{./eps/AMT}
  \caption{Building the AMT from a pattern set.}
\end{figure}

Figure \ref{fig:AMT} shows an example of this transformation. Given a
pattern set $P = \{p_1,\, p_2,\, \dots,\, p_{13}\}$ which will be
transformed to an \textsf{AMT}.  First, since $p_6$ is the shortest
pattern in $P$, we have $lsp = |p_6| = 2$. Then the length-2 prefix of
each pattern is cut off to form a prefix set: $PF = \{p_1^{+2} = aa,\;
p_2^{+2} = aa,\; \dots,\; p_{13}^{+2} = cc\}$.  After removing the
repetitions in $PF$, only 3 distinct prefixes are left: $PF = \{aa,\,
bb,\, cc\}$, and hence $ndp = |PF| = 3$.

Then an index structure (actually the root of the \textsf{AMT}) is
built based on $PF$ to hold its prefixes. Each prefix serves as a
\textsf{key} in the index structure and have a pointer to the child
node (initialized to $NULL$). As will be seen later, the type of the
index structure is selected adaptively according to the characteristic
of $PF$.

% % We use the notation $p_n^{-m}$ to denote the pattern $p_n$ that has
% % its length-$m$ prefix cut off
% While the $key$ component is not just the
% solely prefix string, but the corresponding (partial) structure in the
% node that hold that prefix as a key (which means the $key$ is ``node
% related''). 

After creation of the root, the patterns left in $P$ that have their
length-2 prefixes cut off are divided into sub-pattern-sets: the
patterns with the same length-2 prefixes lost are grouped together to
form a sub-pattern-set. In order to reveal the correspondence between
the sub-pattern-set and its lost prefix, the tuple $(node.key,\;
sub\;pattern\;set)$ is used: the patterns in the $sub$-$pattern$-$set$
have the same prefix $key$ cut off which is held as a key in the newly
created $node$. Using this tuple notation, three tuples are formed
after dividing $P$: $tp_1 = (root.aa,\; SP_1=\{p_1^{-2},\, p_2^{-2},\,
p_3^{-2},\, p_4^{-2},\, p_5^{-2},\, p_7^{-2}\})$,\, $tp_2 =
(root.bb,\; SP_2=\{p_8^{-2},\, p_9^{-2},\, p_{10}^{-2}\})$ and $tp_3 =
(root.cc,\; SP_3=\{p_{11}^{-2},\, p_{12}^{-2},\, p_{13}^{-2}\})$,
which means the pattern set $P$ is divided into three
sub-pattern-sets: $SP_1$, $SP_2$, $SP_3$, and each corresponds to a
prefix key held in the root (it also means that the root will have
three child nodes).

Note that, the shortest pattern $p_6$ has disappeared in the
sub-pattern-sets, since after cutting the length-2 prefix of $p_6$,
there is nothing left. In order to mark the end of a pattern, once the
last part of the pattern has been cut off, that last part is marked
with an asterisk to in the corresponding tree node. It can be seen
that, any path from the root to a key that marked as a pattern end
represents a complete pattern. Therefore, the patterns are stored
implicitly in the \textsf{AMT} and can be reconstructed from paths
ended with an asterisk.

The same procedure is repeated upon each of the sub-pattern-sets in
the tuples. As a further example, $SP_1$ of $tp_1$ is processed. Since
$lsp = |p_7^{-2}| = 6$, the length-$6$ prefix of each pattern in
$SP_1$ is cut off to form $\(PF_1 = \{(p_1^{-2})^{+6},\,
(p_2^{-2})^{+6},\, (p_3^{-2})^{+6},\, (p_4^{-2})^{+6},\,
(p_5^{-2})^{+6},\, (p_7^{-2})^{+6}\}\)$. After discarding the
duplicates in $PF_1$, only three distinct prefixes are left: $PF_1 =
\{a^6,\, e^6,\, b^6\}$ (we use $c^n$ to denote that the character
\emph{c} appears $n$ times).  A new index structure $node_1$ is built
adaptively based on $PF_1$ to hold its prefixes as keys, and then
according to the first component of $tp_1$, $node_1$ is associated
with the key $aa$ in the root ($root.aa$), which makes it to be the
first child of the root. Again the patterns left in $SP_1$ are grouped
based on their lost length-6 prefixes, which formed three new tuples:
$tp_4 = (node_1.a^6,\; SP_4=\{p_1^{-8}\})$, $tp_5 = (node_1.e^6,\;
SP_5=\{p_2^{-8},\, p_3^{-8},\, p_4^{-8}\})$ and $tp_6 = (node_1.b^6,\;
SP_6=\{p_5^{-8}\})$.

During the construction of \textsf{AMT}, a \textsf{breadth-first}
strategy is used to process the sub-pattern-sets and create tree nodes
(which means the next sub-pattern-set to be processed is $SP_2$ rather
than $SP_4$). For the purpose of tracing the order in which the
sub-pattern-sets are processed, a \textsf{first-in-first-out} queue is
employed to maintain the tuples: the tuple at the beginning of the
queue contains the sub-pattern-set that will be processed next, while
the newly created tuples are inserted to the end of the queue in
order. In our example, the root is built initially. Then $tp_1$,
$tp_2$ and $tp_3$ are inserted to the queue. Next $SP_1$ of $tp_1$ is
processed and the newly created tuples $tp_4$, $tp_5$ and $tp_6$ are
inserted to the end of the queue in order. Subsequently, $SP_2$,
$SP_3$, $SP_4$, $SP_5$, $SP_6$, $\dots$ are processed in
sequence. Once there are no tuples left in the queue, the whole
\textsf{AMT} is constructed.

\begin{algorithm}
  \caption{Building the Adaptive Match Tree}
  \label{alg:amt}
  \begin{algorithmic}[1]
    \REQUIRE The pattern set $P$
    \ENSURE The AMT
    \STATE $Q \leftarrow$ create an empty queue
    \STATE $push\_queue((NULL,\,P),\; Q)$
    \STATE
    \WHILE{$Q \neq \emptyset$}
    \STATE $(parent\_node.key,\; P) \leftarrow pop\_queue(Q)$
    \STATE $lsp \leftarrow |p_{st}|$, where $p_{st}$ is the shortest pattern in $P$
    \FOR{each $p \in P$}
    \IF{$|p|=lsp$}
    \STATE mark $p^{+lsp}$ to be the pattern end
    \ENDIF
    \STATE cut off $p^{+lsp}$ of $p$, and put $p^{+lsp}$ to $PF$
    \ENDFOR
    \STATE remove repetitions in $PF$, and let $ndp \leftarrow |PF|$
    \STATE $new\_node \leftarrow$ build an index structure daptively
    based on $PF$
    \IF{$(parent\_node.key = NULL)$}
    \STATE $root \leftarrow new\_node$
    \ELSE
    \STATE associate $new\_node$ with $parent\_node.key$ by a child pointer
    \ENDIF
    \STATE $TP \leftarrow \{(new\_node.pf,\, SP) \mid SP \subseteq P\; and
    \ \forall \ p,\,q \in SP: p,\,q$ have the same length-$lsp$ prefix
    $pf$ cut off\}\
    \FOR{each $tp \in TP$}
    \STATE $push\_queue(Q,\,tp)$
    \ENDFOR
    \ENDWHILE
    \STATE
    \RETURN $root$.
  \end{algorithmic}
\end{algorithm}

The pseudocode of building the \textsf{AMT} from a pattern set is
illustrated in Algorithm \ref{alg:amt}. Firstly, an empty queue $Q$ is
created in line 1 to maintain the tuples. Then the tuple $(NULL, P)$
is inserted to $Q$ in line 2 by the function $push\_queue$ which
always inserts an element to the end of the queue. Since $P$ is the
initial pattern set here, there is no corresponding prefix key of $P$,
and this is indicated by the $NULL$ component in the tuple. By this
way, the creation of the root can be combined with other tree nodes in
the following \textsf{while} loop.

If $Q$ is not empty, the \textsf{while} loop body from line 4 to line
20 constructs a tree node based on the first tuple in $Q$ which is
fetched by the $pop\_queue(Q)$ function in line 5: the sub-pattern-set
on which the tree node is based is assigned to $P$ and the
corresponding key of $P$ in the parent node is denoted by
$parent\_node.key$.  Further the $lsp$ of $P$ is computed in line
6. In the inner \textsf{for} loop from line 7 to line 12, the
length-$lsp$ prefix of each pattern in $P$ is cut off and collected to
form the prefix set $PF$, and if a prefix is the last part of some
pattern, it is marked as a pattern end. Line 13 remove the repetitions
in $PF$, after that a new tree node is created adaptively based on
$PF$ in line 14. The \textsf{if} statement in line 15 determines
whether the newly created node is the root node or not: if the prefix
component of the current tuple is $NULL$, then the new node is the
root node as stated in the last paragraph, otherwise, the new node is
a child node which needs to be associated with the corresponding key
in its parent node. Finally in line 20, the patterns left in $P$ are
divided based on their lost prefixes, and the created tuples are
inserted to the end of the queue in order. Once there is no tuple left
in $Q$, the whole \textsf{AMT} has been completely built, and the root
of \textsf{AMT} is returned.

\subsection{Matching the text}
\label{sec:mp}
 
Once the \textsf{AMT} has been constructed, it is used to search the
patterns in the text string. The algorithm checks every position of
the text string and each position requires a \textsf{matching round}
of the algorithm to verify whether there are some patterns start at
that position (several patterns may start at the same position). For
each position $i$, the sub-string starting at $i$ in the text is
compared segment by segment with the corresponding nodes in
\textsf{AMT} from the root. If the text successfully matched a key
that is marked as a pattern end in some node, the corresponding
pattern is reported to be found at position $i$. However, once there
is a mismatch or the comparing beyond the leaves of \textsf{AMT}, the
current matching round is terminated and the algorithm forwards to the
$i+1$ position of the text to start a new matching round. Whenever
each position of the text has been checked, the whole algorithm is
terminated.

% the algorithm compare the sub-string starting at $i$
% segmentedly with the \textsf{AMT} one node a time from the root

\begin{figure}[htbp]
  \centering
  \includegraphics[width=0.75\textwidth]{./eps/match}
  \caption{Using AMT to locate patterns in the pattern set.}
  \label{fig:2}
\end{figure} 

As an example, given the \textsf{AMT} built in the previous section
and a text string $T$ waiting for match, and assuming the current
matching position is $i$. The matching starts from the root of
\textsf{AMT}: according to $key\_len(root)=2$, the same length
sub-string $T[i,\,2]=aa$ in the text is retrieved as a target among
the keys of root. The searching result is that the target string $aa
\in root$ and it is also marked as a pattern end. Therefore the
pattern $T[i,\,2]=aa$ (actually $p_6$ in $P$) is reported to be found
at position $i$. Then the matching jumps to $node_1$ which is the
child node of $root.aa$.

At $node_1$, according to $key\_len(node_1)=6$, the sub-string
$T[i+2,\,6]=e^6$ is retrieved in $node_1$. It is found that: $e^6 \in
node_1$, but not marked as a pattern end. So the matching immediately
jumps to $node_5$ --the child of $node_1.e^6$, with nothing reported.
 
At $node_5$, since $T[i+8,\,5]=u^5 \in node_5$ and $u^5$ is also
marked as a pattern end, the pattern (actually $p_4$ in $P$)
$T[i,\,13]=a^2e^6u^5$ is reported to be found at $i$. Then the
matching goes to the child of $node_5.u^6$ -- $node_{10}$.

This time, since the corresponding sub-string $T[i+13,\,4]=sscc \notin
node_{10}$, the current matching round at position $i$ is
terminated. The processing moves to the next position $i+1$ and
restart a new matching round from the root. Once all the positions of
the text have been checked, the whole algorithm terminates.

It can be seen that, for each matching round, it corresponds to a
\textsf{matching path} in the \textsf{AMT} that consists of the nodes
on which the comparing has been taken. In this example, the path is:
$root \rightarrow node_1 \rightarrow node_5 \rightarrow node_{10}$,
witch is indicated by the dashed arrows in Figure.

%  % For each node of
% \textsf{AMT}, its inner % keys have the same length, so we use
% $lsp(node)$ to denote the length % of the keys in $node$.

% For each position $i$ of the text, we use
% \textsf{AMT} to check if there are some patterns begins at $i$. The
% checking starts form the root node. First, we look for
% $T[i,len(node)]$ in the keys of the $node$, if there is a match then
% the checking passes to the child node of the corresponding key. Once
% there is mismatch, it turns to the $i+1$ position of text and repeat
% this procedure.



\begin{algorithm}
  \caption{Matching the text}
  \label{alg:mc}
  \begin{algorithmic}[1]
    \REQUIRE ~~\\
    The text string $T$, and the $AMT$ \\
    \ENSURE ~~\\
    The result set $R = \{(i,\,p) \mid 1 \le i \le |T|-lsp+1\ ,\ p \in P :
    T[i,\,|p|] = p \}$
    \STATE $i \leftarrow 1$
    \STATE $m\_len \leftarrow 0$
    \STATE $node \leftarrow root$ of AMT
    \STATE
    \WHILE{$i \le |T|-lsp+1 $}
    %n\STATE $j \leftarrow i$
    \WHILE{$node \neq NULL$ and $\exists key \in node: key =
      T[i+m\_len, \, key\_len(node)]$}
    \STATE $m\_len \leftarrow m\_len + key\_len(node)$
    \IF{$key$ is marked as a pattern end}
    \STATE put $(i,\,T[i,\,m\_len])$ in $R$
    \ENDIF
    \STATE $node \leftarrow$ the child of $node.key$
    %\STATE $j \leftarrow j+node.lsp$
    \ENDWHILE
    \STATE $i \leftarrow i+1$
    \STATE $node \leftarrow root$
    \STATE $m\_len \leftarrow 0$
    \ENDWHILE
    \STATE
    \RETURN $R$
  \end{algorithmic}
\end{algorithm}

The pseudocode of the matching stage of the \textsf{ADMSM} algorithm
is shown in \ref{alg:mc}. $i$ is the current matching position in
text, $m\_len$ is the total length of the currently matched
sub-strings in text in the current matching round and $node$ is the
current node in the \textsf{AMT}. These three variables are initialize
to $1$, $0$ and root of \textsf{AMT} respectively in line 1 to line 3.

The major part of the pseudocode is a double \textsf{while} loop from
line 5 to line 17. The outer \textsf{while} loop checks every position
of the text to find whether there are patterns start at that
position. For each position $i$, the inner \textsf{while} loop
performs a matching round that compares the sub-string starting at $i$
segment by segment with the corresponding nodes in \textsf{AMT}. The
comparing starts form the root: if the corresponding sub-string of $T$
successfully matches some key in the current node, the total matched
length $m\_len$ is increased by the length of the keys in the
node. Additionally, if the matched key is also marked as a pattern
end, the matched pattern $T[i,\,m\_len]$ and its starting position $i$
is inserted into the result set $R$. Next, the comparing transfers to
the child of $node.key$ and makes that child node to be the current
node. Once there is a mismatch or the current node goes beyond the
leaf of \textsf{AMT}(that is $node \neq NULL$), the current matching
round is terminated and a new matching round is performed at the
position $i+1$ of $T$ with the variables $node$ and $m\_len$
reset. Whenever every position of $T$ has been checked, the whole
algorithm is terminated and the result set $R$ is returned.

Note that, since each node in \textsf{AMT} has its own index
structure, the search of the target string among the keys of node in
line 6 uses the specific search routine according to the type of the
index structure.

\subsection{Adaptively build the tree node}
\label{sec:ad}

As we have seen in the preprocessing stage, each node of \textsf{AMT}
is actually an index structure that holds the prefixes in the prefix
set as keys. The type of the index structure is chosen adaptively
according to the characteristic of the prefixes set. Since all the
keys (prefixes) in the prefix set have the same length, the
characteristic of the prefix set can be characterize mainly by two
parameters: the \textsf{length} and \textsf{number} of the keys in the
prefix set, and we use the notations $key\_len$ and $key\_num$ to
denote these two parameters respectively. Note that, these two
parameters are actually the $lsp$ and $ndp$ of the prefix set in the
terms of the previous sections.

In our implementation, three major types of index structures:
\textsf{character map}, \textsf{string array} and \textsf{hash table}
can be selected for prefix sets with different $key\_len$ and
$key\_num$ combinations. These three structures will be then
introduced in details.

\subsubsection{Character map}

If $key\_len=1$, which means all of the keys in the prefixes set are
just single characters, the \textsf{character map} structure
\ref{fig:character map} is adopted as an index structure. Further,
there are four types of character maps with different capacities for
different ranges of $key\_num$ (since $|\Sigma|=256$, there is $1 \le
key\_num \le 256$).

\begin{figure}[htbp]
  \label{fig:character map}
  \centering
  \includegraphics[width=0.8\textwidth]{./eps/Maps}
  \caption{The four types of character maps.}
\end{figure}

The four map types are illustrated in Figure 5 and are named according
to their maximum capacity. Instead of using an array of (key, child
pointer) pairs, the key part and the child pointer part are stored in
separate arrays. This allows to keep the representation compact while
permitting efficient search.

\begin{itemize}
\item \textbf{Map 4} (for $1 \le key\_num \le 4$): The smallest map
type can store up to 4 keys and uses an array of length 4 for keys and
another array of the same length for child pointers. The keys and
pointers are stored at corresponding positions and the keys are sorted
according to their ACSII values. Once the target character is found in
the keys array, its child pointer can be find at the same position in
the pointers array.

\item \textbf{Map 16} (for $5 \le key\_num \le 16$): This map type is
used for storing between 5 and 16 keys. Like Map 4, the keys and
pointers are stored in separate arrays at corresponding positions, but
both arrays have space for 16 entries. A target character can be
retrieved efficiently by a binary search in the key array.

\item \textbf{Map 48} (for $17 \le key\_num \le 48$): As the number of
keys increases, searching the key array becomes expensive. Therefore,
maps with more than 16 keys do not store the keys explicitly. Instead,
a 256-element array is used, which can be indexed with the key byte
(the ASCII value) directly. If a map has between 17 and 48 keys, this
array stores only the \textsf{array indexes} (small integers in the
range of $0 \sim 47$) into a second array which contains up to 48
child pointers. This indirection saves space in comparison to 256
pointers of 8 bytes (assuming in a x86\_64 architecture), because only
one byte is required per array index.

\item \textbf{Map 256} (for $49 \le key\_num \le 256$): The largest
map type is simply an array of 256 pointers and is used for storing
between 49 and 256 keys.  With this representation, the child node can
be found very efficiently using a single lookup of the target byte in
that array.  No additional indirection is necessary. If most entries
are not empty, this representation is also very space efficient
because only pointers need to be stored.

\end{itemize}

The pseudocode of retrieving in a node whose type is character map is
depicted in Figure \ref{fig:character map}.

\begin{algorithm}
  \caption{Retrieving in a node whose type is character map}
  \label{alg:mc}
  \begin{algorithmic}[1]
    \REQUIRE ~~\\
    A tree $node$ whose type is character map\\
    The target character $t\_ch$.
    \ENSURE ~~\\
    The child node of $node.t\_ch$ (possibly NULL).
    \STATE
    \STATE $key\_num \leftarrow$ the number of keys in the character map
    \STATE
    \SWITCH{the type of the character map}
    \CASE{Map 4}
    \FOR{$i = 0$ to $key\_num-1$}
    \IF{$keys[i]=t\_ch$}
    \RETURN $child\_pointers[i]$
    \ENDIF
    \ENDFOR
    \RETURN NULL
    \ENDCASE
    \STATE
    \CASE{Map 16}
    \STATE $low \leftarrow 0, high \leftarrow key\_num-1$
    \WHILE{$low \le high$}
    \STATE $mid = (low+high)/2$
    \IF{$t\_ch=keys[mid]$}
    \RETURN $child\_pointers[mid]$
    \ELSIF{$t\_ch<keys[mid]$}
    \STATE $high=mid-1$
    \ELSE
    \STATE $low=mid+1$
    \ENDIF
    \ENDWHILE
    \RETURN NULL
    \ENDCASE
    \STATE
    \CASE{Map 48}
    \IF{$keys[t\_ch] \neq NULL$}
    \RETURN $child\_pointers[keys[t\_ch]]$
    \ELSE
    \RETURN NULL
    \ENDIF
    \ENDCASE
    \STATE
    \CASE{Map 256}
    \RETURN $child\_pointers[t\_ch]$
    \ENDCASE
    \ENDSWITCH
  \end{algorithmic}
\end{algorithm}

\subsubsection{String array}
\label{sec:string array}

However, if $key\_len > 1$ and $key\_num$ is not very large, the
\textsf{string array} is adopt as the index structure to hold the keys
and their child pointers. Similar to character map, the keys are
stored in lexicographical order in a separate keys array of size
$key\_num \times key\_len$ in which each key takes $key\_len$ bytes,
while the child pointers are stored at the corresponding positions in
another pointers array of size $8 \times key\_num$ bytes. Figure
\ref{fig:string array} depicts a string array structure with keys:
$aaa$, $bbb$ and $ccc$. The keys and pointers array occupies 9 and 24
bytes respectively, and the keys are sorted in lexicographical order.

\begin{figure}[htbp]
  \label{fig:string array}
  \centering
  \includegraphics[width=0.7\textwidth]{./eps/string_array}
  \caption{The string array structure with three keys: $aaa,\, bbb,\, ccc$.}
\end{figure}

For efficiently retrieving, if $key\_num$ is very small, a naive
linear search is used to retrieve the target string in the keys array,
while for a larger $key\_num$, a more efficient (but complicated)
binary search is adopt.  The pseudocode of retrieving in a node whose
type is string array is depicted in Algorithm \ref{alg:string array},
note that, we use the notation $S_i \prec S_j$ to denote that string
$S_i$ is \textsf{lexicographically smaller} than string $S_j$. For the
string array whose $key\_num$ is less than a very small number $M$,
the target string is compared sequentially with the keys in the keys
array, and if the target string is finally found, the child pointer at
the corresponding position in the pointers array is returned,
otherwise, $NULL$ is returned. On the other hand, if $key\_num \geq
M$, the target string is retrieved by a binary search from the middle
key of the keys array. In practice, $M$ takes 5 is a appropriate
choice.

\begin{algorithm}
  \caption{Retrieving in a node whose type is string array}
  \label{alg:string array}
  \begin{algorithmic}[1]
    \REQUIRE ~~\\
    A tree $node$ whose type is string array; \\
    The target string $t\_str$.\\
    \ENSURE ~~\\
    The corresponding child node of $node.t\_str$ (possibly NULL).\\
    \STATE
    \STATE $key\_num \leftarrow$ the number of keys in $node$
    \STATE $key\_len \leftarrow$ the length of keys in $node$
    \STATE 
    \IF{$key\_num < M$}
    \STATE $i \leftarrow 0$
    \WHILE{$i < key\_num$ and $keys[i \cdot key\_len,\, key\_len] \prec
      t\_str$}
    \STATE $i \leftarrow i+1$
    \ENDWHILE
    \IF{$i < key\_num$ and $keys[i \cdot key\_len,\,key\_len]=t\_str$}
    \RETURN $child\_pointers[i]$
    \ELSE
    \RETURN NULL
    \ENDIF
    \ELSE
    \STATE $low \leftarrow 0$, $high \leftarrow key\_num-1$
    \WHILE{$low \le high$}
    \STATE $mid \leftarrow (low+high)/2$
    \IF{$t\_str = keys[mid \cdot key\_len,\, key\_len]$}
    \RETURN $child\_pointers[mid]$
    \ELSIF{$t\_str < keys[mid \cdot key\_len,\, key\_len]$}
    \STATE $high \leftarrow mid - 1$
    \ELSE
    \STATE $low \leftarrow mid + 1$
    \ENDIF
    \ENDWHILE
    \RETURN NULL
    \ENDIF
  \end{algorithmic}
\end{algorithm}

\subsubsection{Hash table}
\label{sec:hash table}

In the case of $key\_len > 1$ and large $key\_num$, retrieving in the
string array structure becomes inefficient even with a binary search,
for this, a more fast (but heavy) index structure --- \textsf{hash
table} is adopted to deal with large volume of keys. In our
implementation, the hash table is an array of child pointers with each
pointer initialized to $NULL$, in addition, a \textsf{string hash
function} is employed. However, building the hash table is somewhat
different from building the other index structures in that the hash
table is built directly based on the sub-pattern-set instead of the
prefix set.

% Indeed, when computing $key\_num$, we
% merely count rather than really constructing the prefix set and then
% computing its $ndp$

Given a sub-pattern-set $SP$, the $key\_len$ of $SP$ is just the $lsp$
of $SP$, while $key\_num$ is the number of the distinct length-$lsp$
prefixes of the patterns in $SP$ ( $key\_num$ is equal to the $ndp$ of
the prefix set that is derived from $SP$).  The size of the hash table
is determined by $key\_num$ and a given \textsf{load factor} (ratio of
keys to slots) $lf$, that is, $table\_size = \lceil key\_num/lf
\rceil$. For example, with the $key\_num$ of 1000 and a load factor of
$70\%$, the table size is $\lceil 1000/0.7 \rceil = 1429$.

Algorithm \ref{alg:hash} shows the pseudocode of building a hash table
based on $SP$. $SP$ is firstly partitioned into small pattern sets by
hashing: for each pattern $p \in SP$, its prefix $p^{+key\_len}$ is
hashed by the string hash function to an integer $i$ between 0 and
$table\_size-1$, and $p$ is associated with the $i$-th slot of the
hash table. After all the patterns have been delivered to the
corresponding slots, the patterns in $SP$ whose length-$key\_len$
prefixes hashing to the same value are grouped together to form a
small pattern set and associated with the corresponding slot of the
hash table. (Note that, we do not cut off the prefixes of the patterns
when partitioning $SP$.) Then, to address the collisions, for each
non-$NULL$ slot of the hash table, a tree node is built based on the
small pattern set associated with that slot as stated before. Hence,
many tree nodes with various index types can be incorporated in one
hash table.

\begin{algorithm}
  \caption{Building a hash table}
  \label{alg:hash}
  \begin{algorithmic}[1]
    \REQUIRE ~~\\
    The sub-pattern-set $SP$  \\
    The load factor $lf$ \\
    \ENSURE ~~\\
    The hash table\\
    \STATE
    \STATE $key\_len \leftarrow lsp$ of $SP$
    \STATE $ken\_num \leftarrow$ the number of distinct length-$lsp$
    prefixes of patterns in $SP$
    \STATE $table\_size \leftarrow \lceil key\_num/lf \rceil$
    \STATE $hash\_table \leftarrow$ creat a hash table of length
    $table\_size$
    \STATE
    \FOR{each $p \in SP$}
    \STATE $i \leftarrow Hash(p^{+key\_len})$
    \STATE associate $p$ with $hash\_table[i]$
    \ENDFOR
    \STATE
    \FOR{$i = 0$ to $table\_size - 1$}
    \IF{$hash\_table[i]\, \neq\, NULL$}
    \STATE $SSP \leftarrow \{p\,|\,p\in SP\; and\; Hash(p^{+key\_len})=i\}$ 
    \STATE $hash\_table[i] \leftarrow$ adaptively build a tree node based on $SSP$ as shown in the
    \textsf{while} loop of Algorithm \ref{alg:amt}
    \ENDIF
    \ENDFOR
    \STATE
    \RETURN $hash\_table$
  \end{algorithmic}
\end{algorithm}

When retrieving a target string in the hash table, the hash value of
that string is used as an index into the table. If the corresponding
slot is $NULL$, which means the target string fails to match any key
in hash table, the algorithm moves immediately to the next position of
the text and restart a new matching round; otherwise, the target is
retrieved again in the child node associated with that slot.

The choice of the hash function can have a significant effect on the
performance of retrieving. In fact, at the beginning of the matching
stage, the hash function is selected randomly from a \textsf{hash
function family $H$}. The hash function adopt in our implementation is
the \textsf{shift-add-xor} hash family from \cite, which is claimed to
be satisfied with the above four properties. The form of the hash
function family is shown in Algorithm \ref{alg:hash fun}.

%  which should satisfy the following properties:

% \begin{itemize}
% \item \textbf{Uniformity.} If a hash function is uniform then the
% probability of an arbitrary key hashing to a given slot is
% $1/table\_size$, independent of the hash values of other keys. In
% practical terms, uniformity means that for a given load factor average
% access time is roughly constant, regardless of table size.

% \item \textbf{Universality.} A hash function family $H$ is universal
% if, for a given $table\_size$ and any pair of valid keys $k_1$ and
% $k_2$, the number of hash functions $h \in H$ such that $h(k_1) =
% h(k_2)$ is less than or equal to $|H|/table\_size$. That is, for a
% randomly-chosen hash function the probability that $k_1$ and $k_2$
% hash to the same value is less than or equal to $1/table\_size$.

% In practice universality means that, with high probability, a
% randomly-chosen hash function will perform well. For any hash function
% it is true that there exist sets of keys that all hash to the same
% value and no hash function is invulnerable to a deliberate attempt to
% identify such a set of keys. However, if a hash function family is
% universal and the functions in the family are uniform then it is
% guaranteed that the family cannot be subjected to such attack.

% \item \textbf{Applicability.} At a more pragmatic level, hash
% functions should be applicable in all circumstances where hashing
% might be used. A function that is limited to a few table sizes, can
% only hash strings of a certain length, is not as valuable as functions
% without such restrictions.

% \item \textbf{Efficiency}. The primary advantage of hashing as an
% access method is its speed: given $ndp$ keys and a table of $O(ndp)$
% size, the search time is $O(1)$, assuming a hash function with time
% complexity $O(1)$. Hash functions should also be small; in many
% applications there is little advantage to a function that is as large
% as the key set.


\begin{algorithm}
  \caption{The \textsf{shift-add-xor} hash family}
  \label{alg:hash fun}
  \begin{algorithmic}[1]
    \REQUIRE ~~\\
    The string $c_1c_2 \dots c_m$  \\
    The $table\_size$  \\
    \ENSURE ~~\\
    The hash value of $c_1c_2 \dots c_m$
    \STATE 
    \STATE $value \leftarrow seed$
    \FOR{$i=1$ to $m$}
    \STATE $value \leftarrow value \oplus (value << L + value >> R + c_i)$
    \ENDFOR
    \RETURN $value$ mod $table\_size$
  \end{algorithmic}
\end{algorithm}

Initially, a randomly positive integer $seed$ is assigned to the hash
$value$ (note that, different seeds give different hash functions and
hence a hash function family is defined). Then the hash value is
computed gradually from each character of the input string by using a
combinations of the \textsf{add, shift, xor} operators: $\oplus$ is
the \textsf{xor} operator, and $value << \;L$ means shift the $value$
to the left $L$ bits, while $c_i$ is the ASCII value of the $i$-th
character of the input string. We use $L=5$ and $R=2$ in our
experiment, which is tested to provide greater likelihood of a uniform
distribution of hash values.

%\end{itemize}

\end{document}



% which means the sub-pattern-set is actually the
% original whole pattern set)

% \begin{tikzpicture}[scale=1.8]

% %\filldraw[fill=red, draw=green] (0,0) circle [ radius=1cm];
% \shade[left color=red, right color=black] (0,0) rectangle (1,1);
% \shadedraw[ball color=yellow, draw=balck] (1.5,0.5) circle [radius=0.5];
% \draw (-1.5,0)--(1.5,0);
% \draw (0,-1.5)--(0,1.5);
% \draw[rotate=45] (0,0) rectangle (0.5,0.5);
% \draw (0,0) rectangle (1,1);
% \fill[green!20!white] (0,0) -- (1cm,0) -- arc[start angle=0, end angle=30, radius=1cm]--(0cm,0cm);
% \draw (-1,-1) rectangle (-0.5,-0.5);
% \draw (3mm,0mm) arc [start angle=45, end angle=110, radiu
% s=3mm];$PF$$PF$$PF$
% \end{tikzpicture}
